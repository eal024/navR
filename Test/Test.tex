% Options for packages loaded elsewhere
\PassOptionsToPackage{unicode}{hyperref}
\PassOptionsToPackage{hyphens}{url}
%
\documentclass[
]{article}
\usepackage{amsmath,amssymb}
\usepackage{lmodern}
\usepackage{ifxetex,ifluatex}
\ifnum 0\ifxetex 1\fi\ifluatex 1\fi=0 % if pdftex
  \usepackage[T1]{fontenc}
  \usepackage[utf8]{inputenc}
  \usepackage{textcomp} % provide euro and other symbols
\else % if luatex or xetex
  \usepackage{unicode-math}
  \defaultfontfeatures{Scale=MatchLowercase}
  \defaultfontfeatures[\rmfamily]{Ligatures=TeX,Scale=1}
\fi
% Use upquote if available, for straight quotes in verbatim environments
\IfFileExists{upquote.sty}{\usepackage{upquote}}{}
\IfFileExists{microtype.sty}{% use microtype if available
  \usepackage[]{microtype}
  \UseMicrotypeSet[protrusion]{basicmath} % disable protrusion for tt fonts
}{}
\makeatletter
\@ifundefined{KOMAClassName}{% if non-KOMA class
  \IfFileExists{parskip.sty}{%
    \usepackage{parskip}
  }{% else
    \setlength{\parindent}{0pt}
    \setlength{\parskip}{6pt plus 2pt minus 1pt}}
}{% if KOMA class
  \KOMAoptions{parskip=half}}
\makeatother
\usepackage{xcolor}
\IfFileExists{xurl.sty}{\usepackage{xurl}}{} % add URL line breaks if available
\IfFileExists{bookmark.sty}{\usepackage{bookmark}}{\usepackage{hyperref}}
\hypersetup{
  pdftitle={Kapittel XXXX},
  pdfauthor={Her skriver du navn.},
  hidelinks,
  pdfcreator={LaTeX via pandoc}}
\urlstyle{same} % disable monospaced font for URLs
\usepackage[margin=1in]{geometry}
\usepackage{graphicx}
\makeatletter
\def\maxwidth{\ifdim\Gin@nat@width>\linewidth\linewidth\else\Gin@nat@width\fi}
\def\maxheight{\ifdim\Gin@nat@height>\textheight\textheight\else\Gin@nat@height\fi}
\makeatother
% Scale images if necessary, so that they will not overflow the page
% margins by default, and it is still possible to overwrite the defaults
% using explicit options in \includegraphics[width, height, ...]{}
\setkeys{Gin}{width=\maxwidth,height=\maxheight,keepaspectratio}
% Set default figure placement to htbp
\makeatletter
\def\fps@figure{htbp}
\makeatother
\setlength{\emergencystretch}{3em} % prevent overfull lines
\providecommand{\tightlist}{%
  \setlength{\itemsep}{0pt}\setlength{\parskip}{0pt}}
\setcounter{secnumdepth}{-\maxdimen} % remove section numbering
\usepackage{booktabs}
\usepackage{longtable}
\usepackage{array}
\usepackage{multirow}
\usepackage{wrapfig}
\usepackage{float}
\usepackage{colortbl}
\usepackage{pdflscape}
\usepackage{tabu}
\usepackage{threeparttable}
\usepackage{threeparttablex}
\usepackage[normalem]{ulem}
\usepackage{makecell}
\usepackage{xcolor}
\ifluatex
  \usepackage{selnolig}  % disable illegal ligatures
\fi

\title{Kapittel XXXX}
\author{Her skriver du navn.}
\date{01 10 2020}

\begin{document}
\maketitle

\hypertarget{post-xx---beskrivelse-av-anslagene}{%
\subsection{Post xx - beskrivelse av
anslagene}\label{post-xx---beskrivelse-av-anslagene}}

\begin{table}[H]
\centering
\begin{tabular}{rrrr}
\toprule
Vedtatt budsjett & Anslag fra 27.7.20 & NAVs nye anslag & Endring fra vedtatt budjsett\\
\midrule
\cellcolor{gray!6}{1684} & \cellcolor{gray!6}{1684} & \cellcolor{gray!6}{1648} & \cellcolor{gray!6}{-36}\\
\bottomrule
\end{tabular}
\end{table}

Arbeids- og velferdsdirektoratet tilrår et anslag for 2020 på 1 648
mill. kroner. Dette er 36 mill. kroner lavere enn forrige anslag og
vedtatt budsjett. For 2021 tilrås et anslag på 1 607 mill. kroner.

De nye anslagene er basert på august-tall.

\hypertarget{regnskapsutvikling}{%
\subsubsection{1.2 Regnskapsutvikling}\label{regnskapsutvikling}}

\begin{table}[H]
\centering
\resizebox{\linewidth}{!}{
\begin{tabular}{lrrrrrrrl}
\toprule
ar & regnskap & Endring & vekst & pris_snitt & Mill. kr i 2020-G & Endring regnskap i 2020-G & vekst faste kr & kategori\\
\midrule
\cellcolor{gray!6}{2016} & \cellcolor{gray!6}{2295} & \cellcolor{gray!6}{-210} & \cellcolor{gray!6}{-0.08} & \cellcolor{gray!6}{91740} & \cellcolor{gray!6}{2523} & \cellcolor{gray!6}{-299} & \cellcolor{gray!6}{-0.11} & \cellcolor{gray!6}{Regnskap}\\
2017 & 2056 & -240 & -0.10 & 93281 & 2223 & -301 & -0.12 & Regnskap\\
\cellcolor{gray!6}{2018} & \cellcolor{gray!6}{1827} & \cellcolor{gray!6}{-229} & \cellcolor{gray!6}{-0.11} & \cellcolor{gray!6}{95800} & \cellcolor{gray!6}{1923} & \cellcolor{gray!6}{-300} & \cellcolor{gray!6}{-0.13} & \cellcolor{gray!6}{Regnskap}\\
2019 & 1709 & -118 & -0.06 & 98866 & 1743 & -180 & -0.09 & Regnskap\\
\cellcolor{gray!6}{2019-08-31} & \cellcolor{gray!6}{1147} & \cellcolor{gray!6}{-88} & \cellcolor{gray!6}{-0.07} & \cellcolor{gray!6}{98370} & \cellcolor{gray!6}{1176} & \cellcolor{gray!6}{-131} & \cellcolor{gray!6}{-0.10} & \cellcolor{gray!6}{Regnskap}\\
\addlinespace
2020-08-31 & 1087 & -60 & -0.05 & 99858 & 1097 & -79 & -0.07 & Regnskap\\
\cellcolor{gray!6}{2020} & \cellcolor{gray!6}{NA} & \cellcolor{gray!6}{-61} & \cellcolor{gray!6}{-0.04} & \cellcolor{gray!6}{100853} & \cellcolor{gray!6}{NA} & \cellcolor{gray!6}{-95} & \cellcolor{gray!6}{-0.05} & \cellcolor{gray!6}{NA}\\
2021 & NA & -41 & -0.02 & 103072 & NA & -75 & -0.05 & \cellcolor{gray!6}{NA}\\
\bottomrule
\end{tabular}}
\end{table}

Her skal det være beskrivende tekst for anslaget.

Her kommer tabellen som viser månedsutvikling. \newpage

\begin{table}
\centering
\resizebox{\linewidth}{!}{
\begin{tabular}{cccc}
\toprule
Antall mnd & \makecell[c]{Prosentvis vekst i \\regnskapstallene målt i fast G} & \makecell[r]{Budsjettanslag for 2020 basert på\\tilsvarende vekst. Mill. kroner} & \makecell[l]{Budsjettanslag for 2021 basert på\\tilsvarende vekst. Mill. kroner}\\
\midrule
\cellcolor{gray!6}{mnd 3} & \cellcolor{gray!6}{-4.57} & \cellcolor{gray!6}{1 724} & \cellcolor{gray!6}{1 645}\\
mnd 6 & -5.61 & 1 705 & 1 610\\
\cellcolor{gray!6}{mnd 9} & \cellcolor{gray!6}{-6.93} & \cellcolor{gray!6}{1 681} & \cellcolor{gray!6}{1 565}\\
mnd 12 & -7.13 & 1 678 & 1 558\\
\bottomrule
\end{tabular}}
\end{table}

\hypertarget{forutsetninger-og-anslag}{%
\subsubsection{1.3 Forutsetninger og
anslag}\label{forutsetninger-og-anslag}}

\end{document}
